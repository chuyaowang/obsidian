\section{Final Report Storyline}\label{final-report-storyline}

\pandocbounded{\includegraphics[keepaspectratio]{Media/Final Report Storyline 2025-05-20 03.33.11.excalidraw.md}}

\subsection{Methods}\label{methods}

\subsubsection{Multiomics data analysis and statistical
testing}\label{multiomics-data-analysis-and-statistical-testing}

For the omics data, preliminary data analysis and visualization were
performed in R 4.4.1. T tests of the transcriptomics data were likely
performed using the DESeq2 package in R, and p values were adjusted
using the Benjamini-Hochberg method. For the proteomics data, median
rather than mean log2 fold changes (log2FC) were used to be more robust
against outliers in proteomics data. Then one sample T tests were
performed with the null hypothesis log2FC equal to 1, implying no change
between the groups. The p values were then adjusted using the
Benjamini-Hochberg method to account for multiple testing.

\subsubsection{Functional annotation}\label{functional-annotation}

Gene Ontology (GO) annotation was performed separately for up-regulated,
down-regulated genes, and up-regulated proteins using ShinyGo 0.82 (add
ref). Protein-protein interaction (PPI) network was constructed
including all significant genes and proteins using the STRING database
(add ref). Analysis and visualization of the network was performed in
Cytoscape 3.10.3 (add ref).

\subsection{Results}\label{results}

\subsubsection{Transcriptomics and proteomics data
distributions}\label{transcriptomics-and-proteomics-data-distributions}

Significantly more genes (3,470) than proteins (679) were detected in
the samples, reflecting differences in the underlying detection methods.
In transcriptomics, sequencers directly output nucleotide sequences that
can be mapped to a reference genome, enabling more comprehensive
detection. In contrast, proteomics relies on inferring protein
identities through database matching of mass and retention time data,
which results in lower detection sensitivity. While neither approach
captures the full transcriptome or proteome, it is important to note
that proteomic data represent a particularly limited subset of the
bacterial proteome, which should be considered when interpreting
significant findings.

For both transcriptomic and proteomic data, most absolute log2FC cluster
around zero (i.e., raw fold change of 1), suggesting a maintained
homeostatic state critical for bacterial survival (Figure X).
Adaptations to phosphate and light levels reflected by larger fold
changes may thus occur to maintain or minimally disrupt this
homeostasis.

The standard error of transcriptomics and proteomics data both inversely
correlate with absolute fold change (Figure X). The distributions show
that smaller fold changes may be more attributed to technical variation
rather than true biological variation.

In the transcriptomic data, fold changes show an inverse correlation
with the mean expression value across samples (\emph{baseMean}) (Figure
X). Genes with higher expression levels tend to exhibit smaller fold
changes, likely because further up-regulation is energetically costly
and may disrupt bacterial homeostasis more significantly.

\subsubsection{Transcriptome and proteome
correlation}\label{transcriptome-and-proteome-correlation}

Overall, there is little correlation between transcriptomic and
proteomic log2FC (Figure XA). Only a small number of genes exhibited
proteomic fold changes that correspond to transcriptomic fold changes.
In particular, after filtering for genes or proteins with absolute
log2FC greater than 1, a subset of genes with either low transcriptomic
log2FC/high proteomic log2FC or vice versa are identified, contracting
expectations of the central dogma (Figure XB). These discrepancies
suggest the presence of intermediate regulatory mechanisms that may be
important for understanding bacterial responses to phosphate conditions
and require further investigation.

\subsubsection{Duplicate translations}\label{duplicate-translations}

Four proteins detected are each translated from two different genes
(Table X). The reasons here, highlighting the need for protein isoform
specification in the database. Interestingly, genes for some of the same
protein do not have close baseMean and log2FC, suggesting more nuance
regulatory mechanisms for protein levels (Figure X).

\begin{longtable}[]{@{}lll@{}}
\toprule\noalign{}
Protein & Gene & Function \\
\midrule\noalign{}
\endhead
\bottomrule\noalign{}
\endlastfoot
Q6YRS9 & slr6016, slr6075 & Unknown \\
P16033 & psbA2, psbA3 & Photosystem II protein D1 \\
P09192 & psbD2, psbD & Photosystem II protein D2 \\
P74750 & DnaE-c, DnaE-n & DNA~polymerase~III~subunit~alpha \\
\end{longtable}

\subsubsection{Investigation of NA ratios in proteomics
data}\label{investigation-of-na-ratios-in-proteomics-data}

The proteomic data include many proteins with missing (NA) log2FC
values. This happens when a protein is not detected in one or both
experimental groups. Since these proteins appear in the dataset, they
were detected in at least one sample. If they are missing in all three
pairs of replicates, there are 3 NA values, making it impossible to
calculate a meaningful fold change between conditions. There are at
least four plausible, non-mutually exclusive explanations for these
missing values:

\begin{enumerate}
\def\labelenumi{\arabic{enumi}.}
\tightlist
\item
  Spurious hit: a fragment from another protein may have been
  incorrectly matched to the protein in question. This can occur due to:

  \begin{enumerate}
  \def\labelenumii{\arabic{enumii}.}
  \tightlist
  \item
    Structural similarity between proteins, causing shared fragments
    with similar retention times.
  \item
    Structural isomers, fragments with the same mass-to-charge ratio
    (m/z) but different structures, being indistinguishable by mass
    spectrometry, especially if they elute at similar times.
  \item
    Low-resolution mass spectrometry that cannot distinguish between
    fragments with very close m/z values (e.g., 70.03 vs.~70.05),
    leading to misidentification.
  \end{enumerate}
\item
  Post-translational modifications or isotope enrichment: these changes
  can alter the mass of protein fragments, preventing correct matching
  in the database.
\item
  Low abundance: the protein may be present at very low levels,
  detectable only in some samples when it happens to exceed the
  detection threshold.
\item
  True biological absence: the protein may be genuinely absent in one
  group and expressed in the other, making it a potentially important
  finding.
\end{enumerate}

If case 4 is true, these NA values could reflect meaningful biological
differences. Visual inspection of the transcriptomic fold changes and p
values for these proteins showed no consistent pattern, suggesting they
are not detected likely due to random variability and not associated
with experimental bias (Figure X). However, without details about the
instrumentation, analysis pipeline, and raw data, no further judgement
on the likelihood of the first 3 cases can be made. The significance of
these proteins across conditions could be further explored by examining
their connections to the identified and quantified differentially
expressed genes and proteins in functional annotations and PPI networks.
However, given the unclear reasons for their non-detection and the
generally weak correlation between transcriptomic and proteomic log2FCs,
any relationships discovered would provide only limited support for more
robust findings and new conclusions drawn from these connections would
have limited reliability.

\subsubsection{Transcriptome and proteome T
test}\label{transcriptome-and-proteome-t-test}

Using a threshold of \textbar log2FC\textbar{} \textgreater{} 1 and
adjusted p value \textless{} 0.05, 153 significantly up-regulated and
120 down-regulated genes were identified (Figure XA). Several adjusted p
values became 0 due to being smaller than the lower limit of floating
point representation and were treated as significant. Using a threshold
of \textbar log2FC\textbar{} \textgreater{} 1 and adjusted p value
\textless{} 0.1, 11 significantly up-regulated genes were identified
(Figure XB). Overall the p values of proteins were higher than those of
genes because the standard error of the proteomics data is much higher,
leading to lower confidence in the fold changes observed (Figure X).

\subsubsection{GO functional annotation}\label{go-functional-annotation}

GO analysis of the up-regulated genes revealed significant enrichment
for phosphate transporter-related terms, indicating increased phosphate
transporter activity in the phosphate-limited group (Figure XA). In
contrast, down-regulated genes were primarily enriched for ribosome and
translation-related GO terms, suggesting a reduction in overall
translational activity under phosphate limitation (Figure XB). GO
analysis of the up-regulated proteins also identified enrichment in
phosphate transporter-related terms, confirming increased phosphate
transporter activity at both the transcriptomic and proteomic levels
(Figure XC). However, only 62\% of up-regulated genes, 80\% of
down-regulated genes, and 82\% of up-regulated proteins were annotated
in the GO database. This incomplete coverage suggests that some relevant
functional annotations may be missing from the analysis.

\subsubsection{Protein-protein interaction network
analysis}\label{protein-protein-interaction-network-analysis}

\paragraph{Clusters identified in
network}\label{clusters-identified-in-network}

Three major clusters are evident in the network (Figure X). Two
correspond to the phosphate transporter and ribosomal genes identified
through GO analysis. The third cluster primarily includes genes and
proteins involved in cell-cell communication, environmental sensing, and
intracellular signaling, most of which are up-regulated. Further
investigation of the genes linking these clusters may uncover key
regulatory relationships and shed light on how these functional modules
interact under phosphate-limited conditions.

\paragraph{Connection between phosphate transporter gene and cell
division
gene}\label{connection-between-phosphate-transporter-gene-and-cell-division-gene}

The reduced cell count observed in the phosphate-limited group may
result from decreased cell division. A network pathway linking the
phosphate transporter cluster to a cell division gene supports this
connection (Figure X). Functional annotations suggest that cellular
anabolic processes mediate the relationship between phosphate
availability and cell division (Table X). This aligns with the earlier
hypothesis that bacteria adjust to environmental conditions in ways that
preserve internal homeostasis.

\begin{longtable}[]{@{}
  >{\raggedright\arraybackslash}p{(\linewidth - 6\tabcolsep) * \real{0.0921}}
  >{\raggedright\arraybackslash}p{(\linewidth - 6\tabcolsep) * \real{0.5000}}
  >{\raggedright\arraybackslash}p{(\linewidth - 6\tabcolsep) * \real{0.1711}}
  >{\raggedright\arraybackslash}p{(\linewidth - 6\tabcolsep) * \real{0.2368}}@{}}
\toprule\noalign{}
\begin{minipage}[b]{\linewidth}\raggedright
Gene
\end{minipage} & \begin{minipage}[b]{\linewidth}\raggedright
Functional annotation
\end{minipage} & \begin{minipage}[b]{\linewidth}\raggedright
Compartment
\end{minipage} & \begin{minipage}[b]{\linewidth}\raggedright
Gene state in data
\end{minipage} \\
\midrule\noalign{}
\endhead
\bottomrule\noalign{}
\endlastfoot
pstA & phosphate transporter & Membrane & Up-regulated \\
sphX & phosphate binding & Membrane & Up-regulated \\
sll0540 & ATP-binding cassette transporter & Membrane & Up-regulated \\
sll0185 & Cellular anabolic processes & Intracellular & Up-regulated \\
slr0373 & Cellular anabolic processes regulation & Intracellular &
Down-regulated \\
slr0374 & Cell division, cellular process & Intracellular &
Down-regulated \\
\end{longtable}

\paragraph{Independent phospholipid synthesis
gene}\label{independent-phospholipid-synthesis-gene}

Because cell growth depends on phospholipid synthesis, we examined the
list of significant genes and proteins with related functional
annotations. We identified \emph{slr0496}, an up-regulated gene involved
in phospholipid synthesis, which is not connected to other nodes in the
PPI network. Although the corresponding protein was not detected, the
gene's up-regulation is consistent with the observed increase in cell
volume under phosphate limitation. Its isolation from the network may
imply the presence of yet unidentified regulatory pathways.

\subsubsection{Hypotheses for observed
changes}\label{hypotheses-for-observed-changes}

In the phosphate-limited group, we observed the following changes:

\begin{enumerate}
\def\labelenumi{\arabic{enumi}.}
\tightlist
\item
  Up-regulation of phosphate transporter expression
\item
  Down-regulation of ribosomal genes and reduced translational activity
\item
  Decreased cell number
\item
  Reduced ploidy
\item
  Increased cell volume
\end{enumerate}

To explain these observations, we propose three hypotheses, illustrated
in Supplementary Figure X. The following assumptions (A) are applied
either universally or selectively across the hypotheses:

\begin{enumerate}
\def\labelenumi{\arabic{enumi}.}
\tightlist
\item
  For all of the hypotheses the underlying assumption is that under
  resource-limited conditions, bacteria prioritize individual survival
  over reproduction to remain competitive.
\item
  In a phosphate limited environment, growth is favored over
  reproduction due to the phosphate requirement for DNA replication.
\item
  More protein needs to be translated during division than growth. Hence
  there is a energy gain when division is stopped.
\item
  There is a physical limit to the number of phosphate transporters that
  can occupy the membrane due to constraints like transporter size,
  lipid packing, and interactions with other membrane components.
\item
  DNA is used as phosphate storage in cyanobacteria.
\end{enumerate}

\textbf{Hypothesis 1: Division halt supports transporter expression}

Under low-phosphate conditions, bacteria increase phosphate transporter
expression (Observation 1). To support this, they halt cell division
(Assumption 2) and redirect the energy saved from reduced protein
synthesis (Assumption 3) toward transporter production. The cessation of
division decreases the need for duplicating DNA and proteins, explaining
Observations 2, 3, and 4. Continued growth without division leads to
increased cell volume (Observation 5).

\textbf{Hypothesis 2: Growth-driven transporter up-regulation}

Phosphate scarcity triggers a shift from division to growth (Assumption
2), resulting in larger cell volume (Observation 5). As the cell
expands, new membrane synthesis naturally includes more phosphate
transporters (Observation 1). The energy saved from suppressed division
(Assumption 3) supports this biosynthesis, which also leads to
Observations 2, 3, and 4, as in Hypothesis 1.

\textbf{Hypothesis 3: Membrane demand drives growth and DNA degradation}

In response to phosphate limitation, bacteria increase phosphate
transporter expression (Observation 1). To accommodate more
transporters, additional membrane area is required (Assumption 4),
driving cell growth and thus Observation 5. To meet the phosphate demand
for new membrane synthesis, DNA is degraded to release phosphate
(Assumption 5), explaining the reduced ploidy (Observation 4). Ongoing
DNA degradation prevents cell division (Observation 3) and reduces the
need for translation of division-related proteins (Observation 2).

\subsection{Follow-up experiments}\label{follow-up-experiments}

\subsubsection{Hypothesis 3}\label{hypothesis-3}

In order to validate hypothesis 3, the assumption that there is an upper
limit to phosphate transporter density on the membrane needs to be first
validated. A modified strain with fluorescently labeled phosphate
transporters can be grown in low and normal phosphate conditions with 4
replicates each like in this study. In addition, an unlabeled control
strain should be grown in the same set-up to confirm fluorescent
labeling does not disrupt normal transporter activity and cell function.
Using fluorescence microscopy combined with cell number and cell size
calculated from the CASY counter, a normalized fluorescence intensity
reflecting phosphate transporter density can be calculated for each
group. If the intensity is approximately equal, it demonstrates that in
low and normal phosphate conditions, the bacteria maintains a similar
phosphate transporter density.

Hypothesis 3 further proposes that the need for more membrane space
promotes cell size increase. Therefore, if the upper limit for phosphate
transporter density can be removed, allowing more phosphate transporters
to be allocated while keeping the membrane area the same, the cell size
increase observed should decrease or diminish. The transporter density
could be altered with two approaches. First, genes that regulate
phosphate transporter density could be identified and knocked out.
However, since it likely involves a complex set of genes, a second
approach that screens or engineers a cyanobacteria strain with naturally
higher maximum phosphate transporter density can be applied. One
possible strain engineering approach is using adaptive laboratory
evolution (add ref,
https://microbialcellfactories.biomedcentral.com/articles/10.1186/1475-2859-12-64).
Cyanobacteria strains would be grown under a low phosphate and low
carbon condition to promote higher transporter expression while limiting
growth. The phosphate transporter density can be dynamically tracked
using fluorescence microscopy. The strain with the highest phosphate
transporter density can then be selected.

Then two groups, control (unmodified) and KO (gene knockout, screened,
or engineered strain), could each be grown in phosphate limited and
normal phosphate conditions with 4 replicates. Their cell number, cell
size, and phosphate transporter density can be quantified and compared
to confirm if the bacteria size growth is the result of needing more
space for phosphate transporters.

\subsection{Discussion}\label{discussion}

\begin{itemize}
\tightlist
\item
  need detailed LC/GC-MS protocol to understand NA ratios
\item
  proteomics data have high standard error, need more sample size and
  tighter control of experimental variability
\item
  need more comprehensive proteomics workflow to capture more proteins
\end{itemize}
