\section{final report draft}\label{final-report-draft}

\subsection{Methods}\label{methods}

\subsubsection{Fluorescence Time-Lapse
Imaging}\label{fluorescence-time-lapse-imaging}

Strain YET911 was tagged with mScarlet (RFP) at the GAL1 locus. Two
pre-cultures---one in 2\% glucose and one in 2\% raffinose---were
transferred to agarose pads containing 0.2\% galactose. The RFP channel
and GFP (background fluorescence control) were imaged every 15 minutes.
Cells are segmented from the resulting images and tracked across frames.
The centroids and mean and standard deviation of fluorescence
intensities were calculated for the identified cells. For analysis, each
transition was aligned at time 0 and truncated to 210 minutes
(glucose→galactose) or 225 minutes (raffinose→galactose) to ensure equal
duration.

\subsubsection{Automated Colony Detection and
Tracking}\label{automated-colony-detection-and-tracking}

Scatter plots were made using the provided cell centroids (x, y) and
areas. OpenCV's contour detection algorithm was applied to identify
globs formed by overlapping circles. Starting colonies were identified
in the first frame. After identifying each colony in the first frame and
calculating its centroid, new cells in subsequent frames were assigned
to colonies by minimizing Euclidean distance to existing centroids.

\subsection{Results}\label{results}

\subsubsection{Slower GAL1 Induction Following Glucose
Pre-Culture}\label{slower-gal1-induction-following-glucose-pre-culture}

Figure X1: Time courses of (A) total cell area, (B) total RFP intensity,
and (C) cell number following a shift from glucose (blue) or raffinose
(red) to galactose.

Figure Supp X1: Correlation of single-cell size and RFP intensity.
Scatter plot of individual cell area versus RFP fluorescence under
raffinose→galactose (A) and glucose→galactose (B) shifts at frame 0.

Figure Supp X2: GFP fluorescence controls. (A,B) Total GFP intensity
over time for raffinose→galactose (left) and glucose→galactose (right).
(C,D) Single-cell GFP traces for the two transitions.

Cells pre-cultured in glucose exhibited a significantly longer plateau
before increasing total area, RFP fluorescence, and cell number, whereas
cells pre-cultured in raffinose began expanding with minimal delay
(Figure X1). Furthermore, although both conditions started with similar
metrics, cells transferring from raffinose yielded more than triple the
total RFP signal than the cells transferring from glucose, despite only
modest increases in area (\textless20\%) and cell count (\textless{}
40\%). This indicates that elevated RFP arises from stronger GAL1
activation and not merely from higher cells numbers. As cell area and
RFP intensity are uncorrelated (Figure Supp X1), greater cell area can
be ruled out the the cause of fluorescence increase. The GFP
fluorescence of raffinose pre-culture was about 20\% higher than that of
glucose pre-culture, likely reflecting its greater cell area and number
(Figure Supp X2A-B).

\subsubsection{Single-Cell Dynamics Reveal Division Patterns and
Induction
Delay}\label{single-cell-dynamics-reveal-division-patterns-and-induction-delay}

Figure X2: Single-cell measurements over time for glucose→galactose
(left) and raffinose→galactose (right): (A, B) cell area traces; (C, D)
RFP intensity traces.

Figure Supp X3. Frequency analysis of cell-division cycles. Power
spectra from Fourier transforms of single-cell area time series in (A)
raffinose→galactose and (B) glucose→galactose transitions.

Single cell area traces show upper size limits that are revisited
periodically by the cells, likely indicative of cell division cycles
(Figure X2A-B). Fourier transforms of the single cell traces reveal
higher frequency oscillations in the raffinose pre-culture condition
(Figure Supp X3), indicating faster cell cycles that lead to its greater
cell number (Figure X1C). Notably, glucose-pre-cultured cells maintained
a larger mean size reflecting faster initial growth rate on glucose
compared with raffinose.

Single cell RFP traces reveal a induction delay in glucose-pre-cultured
cells of about 60 minutes demonstrating catabolite repression, while
almost no delay was observed in the raffinose-pre-cultured cells (Figure
X2C-D). In contrast, GFP remained approximately constant across both
conditions with minor fluctuations, confirming that observed RFP changes
are specific to GAL1 activation (Figure Supp X2C-D).

\subsubsection{Increased Intra- and Inter-Cellular Variability with GAL1
Activation}\label{increased-intra--and-inter-cellular-variability-with-gal1-activation}

Figure X3: Scatter of single-cell RFP standard deviation versus mean for
raffinose→galactose (left) and glucose→galactose (right). Color gradient
indicates cell area.

In both transitions, higher mean RFP correlates with increased standard
deviation, reflecting more uneven distribution of fluorescence signals
within the cells (Figure X3). However, it should be noted that in
smaller cells sub-cellular heterogeneity cannot be resolved due to
resolution limits. Across the population, RFP standard deviation also
broadens with the mean intensity. This inter-cellular heterogeneity is
the result of accumulating stochasticity as more GAL1 gene is expressed
(Figure X3).

\subsubsection{Colony-Level Response
Trajectories}\label{colony-level-response-trajectories}

\begin{quote}
Time series clustering still needs to be done for this part, will lead
to more figures and interpretations.
\end{quote}

Figure X4: Colony-level RFP over time for raffinose→galactose (left) and
glucose→galactose (right). (A,B) average per-cell RFP; (C,D) total RFP.
Color denotes the corresponding metric at time 0.

Figure X5: Colony-level cell area over time for raffinose→galactose
(left) and glucose→galactose (right). (A,B) average per-cell RFP; (C,D)
total RFP. Color denotes the corresponding metric at time 0.

Figure Supp X4: Colony cell-number trajectories. Total cell count versus
time for each colony under (A) raffinose→galactose and (B)
glucose→galactose transitions. Line color reflects the cell count at
time 0.

Colony identification found 60 raffinose pre-culture colonies and 53
glucose pre-culture colonies at time 0. In raffinose, colonies showed
uniform average RFP induction, while in glucose, two clusters of
trajectories were observed between the colonies, reflecting bet-hedging
(Figure X4A--B). Furthermore, the delayed induction observed in Figure
X2D was not randomly distributed in cells but coordinated within the
colonies.

High initial average RFP did not predict final average levels, since
those early high values often arose from single-cell colonies. In
contrast, colonies with high initial total RFP remained at the top,
reflecting that larger colony size produces consistently greater
fluorescence (Figure X4C--D; Supp X4).

Similar trends were observed in cell area: average area rankings shifted
at the end due to single-cell colonies, but total colony area rankings
remained consistent (Figure X5). A slight declining trend were found in
both the average area over time plots. Since the total cell number
increases over time for the colonies (Figure Supp X4), it reveals that
bacteria increase their colony size by producing more but smaller cells
rather than fewer larger cells.

\subsubsection{Division Partitioning Unaffected by Catabolite
Repression}\label{division-partitioning-unaffected-by-catabolite-repression}

\begin{quote}
can do a t test for this to be more robust
\end{quote}

Figure X6: Mean RFP fluorescence over time for (A, B) daughter and (C,
D) mother cells, and (E, F) daughter/mother ratio under
raffinose→galactose (left) and glucose→galactose (right) transitions.

For each frame, daughter cells were defined as cells that newly
appearing cells, and the remaining cells were defined as mother cells.
Both mother and daughter RFP intensities rose over time (Figure X6A--D).
However, the daughter/mother fluorescence ratio remained stable and
comparable across both transitions (Figure X6E--F), indicating that
partitioning mechanisms operate independently of pre-culture conditions,
cell size, division rate, and induction dynamics.
